\documentclass{article}  
\usepackage{ctex} % 引入ctex包处理中文  
\usepackage{fancyhdr} % 引入fancyhdr包来定制页眉和页脚  
\usepackage{refcount} % 引入refcount包来获取标签的页码  
\usepackage{lastpage}

\pagestyle{plain} % 这是默认的页面样式,通常不显示页眉和页脚  
\fancypagestyle{mainmatter}{  
	\fancyhf{} % 清除默认的页眉和页脚设置  
	\lhead{Team \# 2024062828349}  
	\rhead{Page \thepage{} of \pageref{LastPage}} % 显示当前页码和最后一页的页码 
	\fancyfoot[C]{\thepage} 
} 
\begin{document}  
	
	% 设置3号黑体并居中  
	\begin{center}  
		\fontsize{16pt}{19pt}\selectfont \textbf{2024年第三届“钉钉杯”大学生}  
		
		\fontsize{16pt}{19pt}\selectfont  
		\textbf{大数据挑战赛论文}  
		
		% 题目  
		\centering  
		题 目:基于大数据分析的智能推荐系统研究  
	\end{center}  
	
	% 摘要和关键字环境  
	\begin{abstract}  
		本文研究了基于大数据分析的智能推荐系统。首先,我们介绍了推荐系统的基本概念和重要性。然后,我们详细探讨了大数据在推荐系统中的应用,包括数据收集、处理和分析。最后,我们提出了一种新的推荐算法,并进行了实验验证。实验结果表明,我们的算法在推荐准确性和效率方面都有显著提升。  
		
		\textbf{关键字:} 大数据分析,智能推荐系统,推荐算法,实验验证  
	\end{abstract}  
	\newpage  
	
	% 目录  
	\tableofcontents  
	\newpage 
	
	\pagestyle{mainmatter}
	\setcounter{page}{1}
	\section{问题重述}  
	\subsection{问题背景}
	\subsection{数据分析}
	\subsection{问题提出}  
	\section{问题分析}  
	\subsection{问题1分析}
	\subsection{问题2分析}
	\subsection{问题3分析}
	\section{模型假设}  
	\section{定义与符号说明}  
	\section{模型的建立与求解}  
	\subsection{问题1的模型建立与求解}  
	\subsection{问题2的模型建立与求解}  
	\subsection{问题3的模型建立与求解}  
	\section{模型的评价及优化}  
	\subsection{误差分析}  
	\subsection{模型的优点}  
	\subsection{模型的缺点}  
	\subsection{模型的推广} 
	\newpage
	\pagestyle{plain}
	\section{参考文献}
	参考文献内容...  
	\newpage % 新的一页开始  
	\section{附录} 
	附录内容...  
	
	
	
	
\end{document}